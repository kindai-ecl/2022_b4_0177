\documentclass[10pt,twocolumn, a4j]{jsarticle}

\usepackage[dvipdfmx]{graphicx}
\usepackage{amsmath}
\usepackage{eepic}
\usepackage{url}

\setlength{\oddsidemargin}{-12mm}
\setlength{\topmargin}{-16mm}
\setlength{\textwidth}{181mm}
\setlength{\columnsep}{8mm}
\setlength{\textheight}{248.5mm}
\renewcommand{\baselinestretch}{.935}

\pagestyle{empty}


\makeatletter
\def\@biblabel#1{#1)}
\def\@cite#1{#1)}
\makeatother

\begin{document}


\twocolumn[

{\LARGE
\hspace*{17mm}107\hspace*{12mm}
プロセスマイニングを用いた店舗に並ぶ行列の分析
}

\vspace{-2mm}
\begin{flushright}
電子商取引研究室\hspace{1.5zw}李 晃史\hspace{1.5zw}

\end{flushright}
\vspace{3mm}
]

\renewcommand{\thesection}{\arabic{section} .}


\section{序 論}

近年、コロナ禍により飲食店におけるモバイルオーダーを用いたテイクアウトの需要が高まっている
\cite{tyousa1}。

しかし、モバイルオーダーでの注文を受け入れすぎたことによって店舗が混雑になり
販売に支障をきたす事態や、逆に受け入れ制限をしたことにより早く注文を処理できた場合に
本来受け入れることができた注文分の利益を損失する事態が起きている\cite{tyousa2}。

そこで、混雑による支障や利益損失を防ぐために行列を分析して適切な受け入れ人数を設定することが重要であるため行列計測に関して調査した。

研究内容としては店舗で行列の計測を行い、混み具合に適切な受け入れ人数を決定することを
想定して、プロセスマイニングを用いて行列の分析を行った。



\section{プロセスマイニングを用いた行列の分析}

行列の計測は学内にある飲食店で昼食時間(12時〜13時)に実施した。
Expo Goを用いて、行列への参加時・注文開始時・決済開始時・注文完了時・受取時の
5つの区分に分けてを計測した。7月は上記の計測であったが、
12月の計測では上記に加えて男女区分、グループ人数も計測項目に追加した。
計測したデータはプロセスマイニングを用いて分析する。
プロセスマイニングを用いることで、行列の一つの行程にかかった時間や平均時間を抽出することができる。また、一定の時間間隔でのデータやグラフを得ることができる。そのデータやグラフからどの時間帯にどのくらいの混み具合が生じるかを算出することができる。
注文開始から注文完了までの男女差やグループ人数によっての差が生じるのかについても分析した。



\section{結果・考察}
7月に計測した際より計測方法を改善し、店舗で計測を行った結果を表\ref{table1},表\ref{table2}に示す。男女の区分、グループ人数ごとの区分を追加し二日間計測した。


\begin{table}[!htb]
 \begin{center}
  \caption{12/14に計測したデータの平均時間}
   \scalebox{0.65}{
    \begin{tabular}{|c|c|c|c|c|} \hline
     &並び始め→注文開始 & 注文開始→決済 & 決済→注文完了 & 注文完了→商品受取 \\ \hline
全体  & 1分10秒 & 5.6秒 & 14.5秒 & 6分25秒 \\ \hline
男    & 1分57秒 & 4.5秒 & 14.7秒 & 5分35秒 \\ \hline
女    & 1分6秒  & 5.7秒 & 14.4秒 & 6分29秒 \\ \hline
1人   & 56.4秒  & 5.5秒 & 12.5秒 & 6分14秒 \\ \hline
2人   & 1分11秒 & 5.4秒 & 16.8秒 & 6分10秒 \\ \hline
3人   & 1分27秒 & 6.2秒 & 12.8秒 & 7分4秒  \\ \hline
 \end{tabular}
}
 \label{table1}
 \end{center}
\end{table}


\begin{table}[!htb]
 \begin{center}
  \caption{12/16に計測したデータの平均時間}
   \scalebox{0.65}{
    \begin{tabular}{|c|c|c|c|c|} \hline
     &並び始め→注文開始 & 注文開始→決済 & 決済→注文完了 & 注文完了→商品受取 \\ \hline
全体 & 56.4秒  & 5.5秒 & 12.4秒 & 6分14秒 \\ \hline
男   & 1分11秒 & 5.4秒 & 16.8秒 & 6分10秒 \\ \hline
女   & 1分27秒 & 6.2秒 & 12.8秒 & 7分4秒  \\ \hline
1人  & 30.1秒 & 12.1秒 & 10.8秒 & 4分7秒  \\ \hline
2人  & 33.6秒 & 10.3秒 & 22.9秒 & 3分58秒 \\ \hline
3人  & 58.4秒 & 11.8秒 & 16.1秒 & 3分18秒 \\ \hline
4人  & 20.5秒 & 4.5秒  & 17秒   & 5分40秒 \\ \hline
5人  & 1分    & 6.8秒  & 7.6秒  & 4分58秒 \\ \hline
 \end{tabular}
}
 \label{table2}
 \end{center}
\end{table}


本研究で計測した行列の項目において、客の行動によって変動しうるのは
注文開始から注文完了までと考えられるので、
男女の区分、グループ人数ごとの区分を比較をしたが大きい差は見受けられなかった。


\section{結 論}
モバイルオーダーの注文を受け入れる際に混雑による支障や利益損失を防ぐために
適切な受け入れ人数を設定することを想定して、店舗に並ぶ行列をプロセスマイニングを用いて
分析することを提案し、店舗で行列を計測を行い分析を行った。
このように行列を分析することで店舗フローのボトルネックになりうる箇所を特定し、
改善策を打ち出すなど様々に応用できると考える。



\begin{thebibliography}{1}

\bibitem{tyousa1}
  5割以上の飲食店が新型コロナウイルスの影響でテイクアウトを開始/931名に聞いた飲食店の年末の営業と採用活動に関する調査,\url{https://www.foods-ch.com/news/prt_76113}

\bibitem{tyousa2}
  宇都宮陽一,奥田隆史. 多段待ち行列モデルを用いた店舗サービスへのit導入がもたらす影響の分析.情報処理学会研究報告,数理モデル化と問題解決(MPS),2017 

\bibitem{tyousa3}
  オープンソースではじめるプロセスマイニング.ハートコア株式会社,2022


\end{thebibliography}
f




\bibliographystyle{junsrt}
\end{document}
